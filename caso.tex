% This is samplepaper.tex, a sample chapter demonstrating the
% LLNCS macro package for Springer Computer Science proceedings;
% Version 2.20 of 2017/10/04
%
\documentclass[runningheads]{llncs}
%
\usepackage{graphicx}
\usepackage{amssymb}
\usepackage{amsmath}
\usepackage{siunitx}
\usepackage{booktabs}
\usepackage{float}
\usepackage[spanish]{babel}
\usepackage[utf8]{inputenc}
\usepackage[T1]{fontenc}

% Used for displaying a sample figure. If possible, figure files should
% be included in EPS format.
%
% If you use the hyperref package, please uncomment the following line
% to display URLs in blue roman font according to Springer's eBook style:
% \renewcommand\UrlFont{\color{blue}\rmfamily}

\begin{document}

\title{Paper Title}
%If Title is too long, use \titlerunning
%\titlerunning{Short Title}

%Single insitute
\author{Diego Lupi\and Pedro Nieto\and Huaira Gómez}
%If there are too many authors, use \authorrunning
%\authorrunning{First Author et al.}
\institute{FaMAF - Universidad Nacional de Córdoba, Córdoba, Argentina}


%% Multiple insitutes - ALTERNATIVE to the above
% \author{%
%     Firstname Lastname\inst{1} \and
%     Firstname Lastname\inst{2}
% }
%
%If there are too many authors, use \authorrunning
%  \authorrunning{First Author et al.}
%
%  \institute{
%      Insitute 1\\
%      \email{...}\and
%      Insitute 2\\
%      \email{...}
%}

\maketitle

\begin{abstract}
Easycrypt\cite{ref_article1} es una herramienta automatizada que soporta la construcción y verificación de pruebas de seguridad de sistemas criptográficos. Permite mejorar la confianza en sistemas criptográficos mediante la entrega de pruebas verificadas formalmente que resultan en sus metas propuestas. Provee una plataforma versátil que soporta pruebas automatizadas pero también permite al usuario realizar puebas complejas de manera interactiva entrelazando la verificación del programa con la formalización de las matemáticas, hecho fundamental al formalizar pruebas criptográficas. Con este paper nos proponemos mostrar las caracteristicas de esta herramienta y compararla con herramientas similares.

\keywords{Easycrypt  \and Game-based cryptographic proofs \and Probabilistic.}
\end{abstract}
%
%
%
\section{Introducción}
Desde siempre las pruebas criptograficas fueron propensas a errores, lo que naturalmente las puede llevar a ser erróneas.
 En particular en las pruebas de seguridad criptograficas la correctitud es critica para mejorar la confianza en el sistema criptografico. Actualmente se tiende a generar mas pruebas de seguridad de las que se pueden verificar y se omiten detalles finos desde un analisis formal que pueden tener grandes efectos en la practica. Teniendo en cuenta que los sistemas criptograficos en el mundo real pueden ser vulnerados, es necesario hacer las verificaciones sobre los pruebas de los sistemas criptograficos para evitar un desastre en el area de la seguridad.

Easycrypt es una herramienta automatizada que permite la construccion de pruebas de seguridad de sistemas criptograficos y su verificacion de manera interactiva usando la secuencialidad del codigo con un enfoque de game-based cryptographic proofs. Este enfoque consiste en la interaccion de un retador y un adversario, donde se especifica explicitamente la meta que adversario intenta alcanzar, como por ejemplo suponer de manera correcta una porcion de informacion oculta. En Easycrypt los juegos criptograficos se modelan como modulos, que consisten en procedimientos escritos en lenguaje propio de la herramienta. Por otra parte los adversarios se modelan como modulos abstractos, modulos cuyo codigo es desconocido y puede cuantificarse.

 Posteriormente se sumo al desarrollo la École Polytechnique (Escuela Politecnica). IMDEA software institute es un instituto para el estudio avanzado de tecnologias para el desarrollo de software asentado en Madrid, España. Inria es un centro de investigación francés especializado en Ciencias de la Computación, teoría de control y matemáticas aplicadas. Por ultimo, la École Polytechnique es una gran escuela de ingenieros francesa bajo la tutela del Ministerio de Defensa.

El primer prototipo de EasyCrypt lanzado en 2009 fue desarrollado por IMDEA Software Institute, e Inria. Constaba de una interfaz de linea de comando y funcionalidades muy acotadas. Posteriormente se sumo al desarrollo la École Polytechnique (Escuela Politecnica). IMDEA software institute es un instituto para el estudio avanzado de tecnologias para el desarrollo de software asentado en Madrid, España. Inria es un centro de investigación francés especializado en Ciencias de la Computación, teoría de control y matemáticas aplicadas. Por ultimo, la École Polytechnique es una gran escuela de ingenieros francesa bajo la tutela del Ministerio de Defensa. En el año 2012 se le hizo una reimplementacion completa al prototipo con el objetivo de superar varias de las limitaciones que este revelo. Actualmente se encuentra en la version 1.0 que fue liberada el 10 Octubre de 2017. En esta version los desarrolladores permitieron que EasyCrypt pueda ejecutar scripts interactivamente en Proof General\cite{ref_webpage1}, dandole a la herramienta una interfaz grafica interactiva en la que el usuario puede simular paso a paso la verificacion de su especificacion, otorgando la posibilidad al usuario de elegir el enfoque por el cual quiere verificar la misma. Por otro lado para proveer las bases requeridas para llevar a cabo algunos razonamientos criptograficos estandares se implementaron cuatro logicas, lo que permite realizar pruebas mas complejas, que en versiones anteriores no eran verificables.

\section{Caracteristicas de EasyCrypt}

EasyCrypt esta diseñada para verificar pruebas criptograficas de manera estructurada. La herramienta puede ayudar a corregir errores y obtener la seguridad probable de sistemas criptograficos. Estos sistemas practican la comunicacion segura, ante la presencia de terceros, en los ambitos de comercio electronico, crypto-monedas, claves de computadoras, tarjetas de pagos con chips y comunicaciones militares. 

La herramienta permite codificar y verificar game-based proofs, pero tiene distintos lenguajes para distintas tareas. El principal lenguaje de especificacion de EasyCrypt es el lenguaje de expresiones, en el cual se definen los tipos junto con los operadores que se pueden ser aplicados. Este lenguaje soporta el polimorfismo parametrico. Por otra parte, los lenguajes de expresiones no son adecuados para definir juegos y otras estructuras de datos como esquemas criptograficos y oraculos, debido a la naturaleza dependiente del estado previo de los algoritmos secuenciales. Por eso EasyCrypt usa un lenguaje diferente, llamado pWhile\cite{ref_book1} (probabilistic while) para definirlos:

\begin{table}[H]
  \setlength{\tabcolsep}{12pt}
  \caption{Lenguaje pWhile}
  \label{tab:simple}
  \centering
  \begin{tabular}{ll}
    \toprule
    $\mathcal{C}$ ::= skip & nop\\
    \hspace{0.5cm}| $\mathcal{V}$	$\xleftarrow{}$ $\mathcal{E}$ & assignment\\
    \hspace{0.5cm}| $\mathcal{V}$ $\xleftarrow{\text{\textdollar}}$ $\mathcal{D}$$\mathcal{E}$ & random sampling\\
    \hspace{0.5cm}| if $\mathcal{E}$ then $\mathcal{C}$ else $\mathcal{C}$ & conditional\\
    \hspace{0.5cm}| while $\mathcal{E}$ do $\mathcal{C}$ & while loop\\
    \hspace{0.5cm}| $\mathcal{V}$	$\xleftarrow{}$ $\mathcal{P}$($\mathcal{E}$,...,$\mathcal{E}$) & procedure call\\
    \hspace{0.5cm}| $\mathcal{C}$; $\mathcal{C}$ & sequence\\
    \bottomrule
  \end{tabular}
\end{table}


La herramienta se restringe a la etapa de verificacion del desarrollo de software. En el trabajo Mind the Gap: Modular Machine-checked Proofs of One-Round Key Exchange Protocols\cite{ref_article2} se desarrolla una nueva prueba de seguridad generia para protocolos intercambio de claves, y se lo instancia para obtener pruebas de seguridad de protocolos conocidos respecto a distintos modelos de adversarios usando EasyCrypt.

\subsection{Aspectos tecnicos}

EasyCrypt esta centrado en el enfoque game-based consiste en la interaccion de un retador y un adversario, donde se especifica explicitamente la meta que adversario intenta alcanzar, como por ejemplo suponer de manera correcta una porcion de informacion oculta. Luego, para representar estos modelos en memoria usa modulos los juegos, que consisten en procedimientos escritos en un lenguaje imperativo que soportan ciclos y operaciones de muestreo aleatorio y son representados como modulos abstactos—modulos cuyo codigo es desconocido, para los adversarios.

A su vez, la herramienta tiene cuatro logicas, la logica probabilistica, relacional de Hoare (pRHL), que relaciona pares de procedimientos; una logica probabilistica de Hoare (pHL), que permite probar la probabilidad de que una post-condicion se mantenga luego de la ejecucion de un procedimiento; una logica posibilistica de Hoare (HL); y una logica ambiental de alto orden para probar hechos matematicos y conectar los juicios de las otras logicas. Una vez que se expresaron las metas, las pruebas son llevadas a cabo usando tacticas, las cuales permiten transformar los lemas expresados en cero o mas sublemas—con condiciones suficientes para que satisfaga el lema original. Luego los sublemas suficientemente simples pueden ser probados por SMT solvers.

\subsection{Usabilidad}
EasyCrypt se puede utilizar mediante lineas de comando o mediante una interfaz grafica interactiva la cual se ejecuta sobre Emacs\cite{ref_webpage2}. Para correrlo mediante lineas de comando se necesita pasarle como argumento el archivo del modelo que debe ser del formato \textit{file.ec}, luego se puede avanzar sobre la ejecucion del codigo o aplicar tacticas mediante nuevos comandos. Por otra parte la ejecucion mediante la interfaz grafica requiere ejecutar Emacs y abrir un archivo de formato \textit{file.ec} y automaticamente se carga una nueva interfaz que permite avanzar en paso a paso sobre la ejecucion del codigo y aplicar tacticas muy facilmente. Ademas en la ultima version la herramienta esta disponible en MacOSX, Windows y las distribuciones de Linux mas conocidas.

\section{Comparación con otras herramientas}
Como se nombró anteriormente Easycrypt es una herramienta que se utiliza para la verificación de pruebas de seguridad de sistemas criptográficos, pero no es la única. Para saber sus limitaciones y sus extensiones se comparó con otras herramientas(CertiCrypt, Verypto).

 Para este tipo de herramienta la legibilidad del codigo es fundamental ya que se busca qur los criptografos puedan comprenderlas y evaluarlas. Ademas, la sintaxis intuitiva apoya a los usuarios durante la formalización, ya que pueden enfocarse en formalizar los argumentos en lugar de que tratar de entender programas poco legibles. Todas las herramientas antes mencionadas logran buena legibilidad, pero de diferentes maneras. En EasyCrypt y CertiCrypt se utiliza un lenguaje de procedimiento imperativo en sus lógicas. Ellos modelan de cerca la idea de Bellare y Rogaway de un lenguaje con estado con oráculos como procedimientos\cite{ref_article3}. Verypto integra profundamente un lenguaje de orden superior con referencias mutables basadas en índices de Bruijn en HOL\cite{ref_article4}. La legibilidad se recupera al reflejar la sintaxis en el lenguaje de términos de HOL mediante el análisis y pretty-printing tricks.

 Otra caracteristica importante es la expresividad, la misma tiene dos dimensiones, la sintaxis y el dominio semántico. Si bien EasyCrypt y CertiCrypt admiten distribuciones discretas de subprobabilidad y un lenguaje de procedimiento, EasyCrypt también proporciona un sistema modular para admitir la abstracción y la reutilización. Verypto provee un marco más general, ya que se basa en la teoría de medidas y, por lo tanto, admite distribuciones continuas y funciones de orden superior.
 
EasyCrypt es presentado como el sucesor de CertiCrypt ya que  presenta un mecanismo para compilar pruebas de CertiCrypt en EasyCrypt. Ademas en EasyCrypt el esfuerzo humano es menor como se ve la fig.1 donde se compara CertiCrypt con EasyCrypt en varias pruebas de seguridad formalizadas en ambos sistemas. Los tiempos se miden en una Intel Core 2 Duo con 2.8GHz y 4GB de RAM sobre Mac OS X 10.6.7.



\begin{tabular}{ |p{3cm}|p{3cm}|p{3cm}|p{3cm}|p{3cm}|  }
 \hline
 casos de estudio & \multicolumn{2}{|c|}{CertiCrypt} & \multicolumn{2}{|c|}{EasyCrypt} \\
 
 &Lines&Time&Lines&Time\\
 \hline
 ElGamal(IND-CPA)   & 565    &45s &190 &12s \\
 Hashed ElGamal(IND-CPA)& 1255  & 1m 5s & 243  & 33s\\
 Full-Domain Hash(EF-CMA) &2035 & 5m 46s&  509 & 1m 26s\\
 Cramer-Shoup(IND-CCA)    &n/a & n/a& 1637 &5m 12s\\
 \hline
\end{tabular}



\section{Casos de estudio de EasyCript}
EasyCrypt es util en sistemas de seguridad. Estos sistemas tienen una fuerte dependencia en la criptografia para garantizar las propiedades deseadas de los mismos. Por lo tanto para que estos sistemas sean correctos (seguros) es necesario contar con una demostracion de que los esquemas criptograficos utilizados por el sistema lo son. Pero estas demostraciones pueden ser propensas a errores, por lo que tambien deben ser provadas correctas, y este es el trabajo de EasyCript. Acontinuacion se pasa a mostrar dos ejemplos de la utilidad e importancia de la herramienta.

\textbf{CMAC y sus variantes\cite{ref_article5}:}
El estandar CMAC cuando fue propuesto inicialmente por Iwata y Kurosawa como OMAC1 fue acompañada por una compleja prueba de seguridad basada en juego. Se utilizo EasyCrypt para formalizar una prueba de \textit{unforgeability} para CMAC. Para esto fue necesario una mejora y extencion a la libreria estandar de EasyCrypt. Ademas de demostrar la seguridad similar a la obtenida por Iwata y Kurosawa\cite{ref_article6} en CMAC, se probo la seguridad de construcciones intermedias como ECBC, FCBC y XCBC\cite{ref_article7}.

\textbf{Prueba de \textbf{IND-CPA security} en Hashed Elgamal:}
La seguridad de hashed Elgamal puede ser reducida a la suposición de \textbf{Computational Diffie-Hellman} (CDH). Esta es la supocicion de que es dificil computar \begin{equation} g^{xy} \end{equation} teniendo \begin{equation} g^{x} \end{equation} y \begin{equation} g^{y} \end{equation} donde x e y son elementos random desitribuidos uniformemente  en $\mathbb{Z}_q$. Esta prueba consta de 243 lineas de codigo y tardó 33s. Los tiempos fueron medidos en un procesador 2.8GHz Intel Core 2 Duo con 4GB de memoria RAM bajo un sistema operativo Mac OS X 10.6.7.

\section{Caso de estudio: Demostracion formal de CMAC y sus variantes}
\textbf{Introduccion.} Un codigo de autentificacion de mensajes(MAC) es una pequeña porcion de informacion (\textit{tag}) que se agrega a un mensaje y se usa para verificar su autenticidad e integridad. Una MAC se considera segura cuando un a esquema de algoritmos generadores de \textit{tags} de MACs le resulta extremadamente dificil producir un \textit{tag} valido sin conocer la clave secreta, aun teniendo la habilidad de obtener \textit{tags} validos de mensajes especificos. CMAC es actualmente el estandar MAC, extensamente usado en la practica. Esta basado en \textit{cypher block chaining MAC} (CBC-MAC), el cual es un esquema muy sencillo que permite la cronstruccion de un esquema MAC para un mensaje de tamaño fijo que se restringe a partir del tamaño de bloque de un \textit{block cypher} que se asume seguro. Para sobreponerse a esta restricción se propusieron variantes de CBC-MAC llamadas ECBC, FCBC, XCBC. Sin embargo, cada uno de estos esquemas requiere una clave mas larga que la solicitada por el \textit{block cypher}. Por otra parte CMAC fue la primer variante \textit{one-key CBC-MAC}, solo usa clave de \textit{block cypher} y fue probado seguro por Iwata y Kurosawa.

Para comprender el mecanismo de CMAC es necesario entender algunos conceptos escenciales de MAC y \textit{block cypher}. Primero, un esquema MAC se compone de tres algoritmos, un algoritmo probabilistico generador de claves, que para la entrada de parámetros de seguridad devuelve una clave \textit{k} $\in$ $\mathcal{K}$ \textit{keyspace}; un algoritmo de \textit{tagging} que dado una clave y un mensaje, genera un \textit{tag}; y un algoritmo de verificación, que dado una clave, un mensaje, y un \textit{tag}, devuelve un booleano que identifica si el \textit{tag} es valido para el mensaje con esa clave. Segundo, un \textit{block cypher} es una familia de permutaciones sobre un conjunto finito de bloques indexado por $\mathcal{K}$.

Para llevar a cabo esta demostración se usaran juegos, algoritmos probabilisticos que describen como puede interactuar un adversario con el esquema que se está tratando de romper su seguridad, y en que consiste dicho quiebre. Entonces la seguridad queda expresada como una propiedad de la probabilidad de que ocurra una violación de seguridad de cualquier adversario, dentro de una clase admisible de adversarios. 

Ahora que tenemos estos conceptos mas claros nos podemos centrar en el uso de EasyCript para asistir en la demostración.

%En el area de estudio de la cryptografia una nocion importante a la hora de diseñar un esquema criptografico es el de indistinguibilidad respecto a una \textit{funcion random}. 



\begin{thebibliography}{8}
\bibitem{ref_article1}
Gilles Barthe, Juan Manuel Crespo, Benjamin Gregoire, Cesar Kunz, Santiago Zanella Beguelin. Computer-Aided Cryptographic Proofs. Third International Conference, 2012.
\bibitem{ref_webpage1}
OCaml Website: (2013) https://ocaml.org.
\bibitem{ref_book1}
G. Barthe, B. Grégoire, and S. Zanella Béguelin, “Probabilistic relational hoare
logics for computer-aided security proofs,” in Mathematics of Program Construction
(J. Gibbons and P. Nogueira, eds.), vol. 7342 of Lecture Notes in
Computer Science, pp. 1–6, Springer Berlin Heidelberg, 2012.
\bibitem{ref_article2}
Gilles Barthe, Juan Manuel Crespo, Yassine Lakhnech, Benedikt Schmidt. Mind the Gap: Modular Machine-checked Proofs of One-Round Key Exchange Protocols. 34th Annual International Conference on the Theory and Applications of Cryptographic Techniques, 2015.
\bibitem{ref_webpage2}
Proof-General Website: (2016) https://proofgeneral.github.io.
\bibitem{ref_article3}
Bellare, M., Rogaway, P.: The security of triple encryption and a framework for
code-based game-playing proofs. In: EUROCRYPT 2006. LNCS, vol. 4004, pp.
409–426. Springer (2006)
\bibitem{ref_article4}
Backes, M., Berg, M., Unruh, D.: A formal language for cryptographic pseudocode.
In: LPAR 2008. LNCS, vol. 5330, pp. 353–376. Springer (2008)
\bibitem{ref_article5}
Cécile Baritel-Ruet, François Dupressoir, Pierre-Alain Fouque and Benjamin Grégoire: Formal Security Proof of CMAC and its Variants.
31st IEEE Computer Security Foundations Symposium, 2018.
\bibitem{ref_article6}
Tetsu Iwata and Kaoru Kurosawa. OMAC: One-key CBC MAC. In FSE, volume 2887, pages 129–153. Springer, 2003.
\bibitem{ref_article7}
John Black and Phillip Rogaway. CBC MACs for arbitrary-length messages: The three-key constructions. Journal of Cryptology, 18(2):111–131, 2005
\end{thebibliography}
\end{document}
\grid


